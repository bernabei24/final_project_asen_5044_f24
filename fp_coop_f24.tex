  %%%%%%%%%%%%%%%%%%%%%%%%%%%%%%%%%%%%%%%%%

% Short Sectioned Assignment
% LaTeX Template
% Version 1.0 (5/5/12)
%
% This template has been downloaded from:
% http://www.LaTeXTemplates.com
%
% Original author:
% Frits Wenneker (http://www.howtotex.com)
%
% License:
% CC BY-NC-SA 3.0 (http://creativecommons.org/licenses/by-nc-sa/3.0/)
%
%%%%%%%%%%%%%%%%%%%%%%%%%%%%%%%%%%%%%%%%%

%----------------------------------------------------------------------------------------
%   PACKAGES AND OTHER DOCUMENT CONFIGURATIONS
%----------------------------------------------------------------------------------------

\documentclass[paper=a4, fontsize=11pt]{scrartcl} % A4 paper and 11pt font size

\usepackage[T1]{fontenc} % Use 8-bit encoding that has 256 glyphs
\usepackage{palatino} % Use the Adobe Utopia font for the document - comment this line to return to the LaTeX default
\usepackage[english]{babel} % English language/hyphenation
\usepackage{amsmath,amsfonts,amsthm} % Math packages
\usepackage{multicol,lastpage,fullpage,framed,fancybox,enumerate,tikz}
\usepackage{lipsum} % Used for inserting dummy 'Lorem ipsum' text into the template
\usepackage{mathrsfs}
\usepackage{graphicx}
\usepackage{mathtools}
%
%
%
%   A T T E N T I O N ! ! !
%
%   SET YOUR GRAPHICS FOLDER IN THE LINE BELOW
%
\graphicspath{ {.} }
\usepackage{sectsty} % Allows customizing section commands
\allsectionsfont{\centering \normalfont\scshape} % Make all sections centered, the default font and small caps

\usepackage{fancyhdr} % Custom headers and footers
\pagestyle{fancy plain} % Makes all pages in the document conform to the custom headers and footers
\fancyhead[L]{\textsc{CSEN 5830}}
\fancyhead[R]{\textsc{Homework #3}} % No page header - if you want one, create it in the same way as the footers below
\fancyfoot[L]{} % Empty left footer
\fancyfoot[C]{} % Empty center footer
\fancyfoot[C]{- \thepage -} % Page numbering for right footer
\renewcommand{\headrulewidth}{0.5pt} % Remove header underlines
\renewcommand{\footrulewidth}{0pt} % Remove footer underlines
\setlength{\headheight}{13.6pt} % Customize the height of the header

\fancypagestyle{noheader}{    %create style that allows to skip header manually on pages with new section
    \fancyhead{}
    \renewcommand{\headrulewidth}{0pt}
}

\numberwithin{equation}{section} % Number equations within sections (i.e. 1.1, 1.2, 2.1, 2.2 instead of 1, 2, 3, 4)
\numberwithin{figure}{section} % Number figures within sections (i.e. 1.1, 1.2, 2.1, 2.2 instead of 1, 2, 3, 4)
\numberwithin{table}{section} % Number tables within sections (i.e. 1.1, 1.2, 2.1, 2.2 instead of 1, 2, 3, 4)

\setlength\parindent{0pt} % Removes all indentation from paragraphs - comment this line for an assignment with lots of text

%----------------------------------------------------------------------------------------
%   TITLE SECTION
%----------------------------------------------------------------------------------------

\newcommand{\horrule}[1]{\rule{\linewidth}{#1}} % Create horizontal rule command with 1 argument of height

\title{
\normalfont \LARGE
\textsc{University of Colorado at Boulder} \\ [25pt] % Your university, school and/or department name(s)
\textsc{ASEN 5044 - Statistical State Estimation for Dynamical Systems} \\ [20pt]
\textsc{Fall 2024} \\ [20pt]
\textsc{Professor: Dr. Nisar Ahmed} \\ [12pt]
\horrule{1pt} \\[0.4cm] % Thin top horizontal rule
\huge Final Project \\ % The assignment title
\huge (Cooperative Air-Ground Robot Localization) \\ 
\horrule{1pt} \\[0.6cm] % Thick bottom horizontal rule
}

\author{
  \textsc{ Team Members:} \\ [4 mm]
  \textsc{ Nicholas Martinez}\\[2mm]
  \textsc{ Whit Whittall } \\[2mm]
  \textsc{ Michael Bernabei}\\[2mm]
}

\date{\normalsize\today} % Today's date or a custom date

\begin{document}

\maketitle % Print the title
\thispagestyle{empty} %make title page header empty
\newpage


%----------------------------------------------------------------------------------------
%   PROBLEM SECTION
%----------------------------------------------------------------------------------------
%

\textbf{\\ Part I: Deterministic System Analysis }
\begin{framed}
\textbf{Part 1.} \\

We are given the Equation Of Motion (EOM) for the Unmanned Ground Vehicle (UGV).  The EOM is,

\begin{align*}
    \dot{\xi_g} &= v_g \cos{\theta_g} + \Tilde{w}_{x,g} \\
    \dot{\eta_g} &= v_g \sin{\theta_g} + \Tilde{w}_{y,g} \\
    \dot{\theta_g} &= \frac{v_g}{L}\tan{\phi_g}  + \Tilde{w}_{\omega,g} \\
\end{align*}

and for the Unmanned Aerial Vehicle (UAV) we have the following EOM,

\begin{align*}
    \dot{\xi_a} &= v_a \cos{\theta_a} + \Tilde{w}_{x,a} \\
    \dot{\eta_a} &= v_a \sin{\theta_a} + \Tilde{w}_{y,a} \\
    \dot{\theta_a} &= \omega_a  + \Tilde{w}_{\omega,a} \\
\end{align*}

where $\Tilde{w}_a = [ \Tilde{w}_{x,a},  \Tilde{w}_{y,a},  \Tilde{w}_{\omega,a}]^T$ and $\Tilde{w}_g = [ \Tilde{w}_{x,g},  \Tilde{w}_{y,g},  \Tilde{w}_{\omega,g}]^T$ are the process noise for the UAV And UGV respectively.  We are also given the following sensing model,

\begin{align*}
y(t) &=  
\begin{bmatrix} \arctan{(\frac{\eta_a - \eta_g}{\xi_a - \xi_g}}) - \theta_g   \\ 
                \sqrt{  (\xi_g - \xi_a)^2 + (\eta_g - \eta_a)^2}  \\
                \arctan{(\frac{\eta_g - \eta_a}{\xi_g - \xi_a})} - \theta_a \\
                \xi_a \\
                \eta_a
\end{bmatrix}  + \Tilde{\bold{v}}(t) \\
\end{align*}

where $\Tilde{\bold{v}}(t) \in \mathbb{R}^5$ is the sensor error vector.  Finally, we are given the combined states, control inputs, and disturbance inputs as,

\begin{align*}
    \bold{x}(t) &= [ \xi_g \ \ \ \eta_g \ \ \ \theta_g \ \ \ \xi_a \ \ \ \eta_a \ \ \ \theta_a ] ^T ,\\
    \bold{u}(t) &= [ \bold{u}_g \ \ \bold{u}_a ]^T ,\\
    \bold{\Tilde{w}}(t) &= [\bold{\Tilde{w}_g \ \ \bold{\Tilde{w}}_a}]^T \\
\end{align*}

The state is $x = [\xi_g \ \  \eta_g \ \  \theta_g \ \  \xi_a \ \  \eta_a \ \ \theta_a ]^T = [ x_1 \ \ x_2 \  x_3 \ \ x_4 \ \ x_5 \ \ x_6 ]^T$ and our inputs $u = [ \bold{u}_g \ \ \bold{u}_a ]^T = [v_g \ \ \phi_g \ \ v_a \ \ \phi_a]^T = [u_1 \ \ u_2 \ \ u_3 \ \ u_4 ]^T$.  We then have the following after substituting in our state and input variables,

\begin{align*}
    \dot{x} &= \begin{bmatrix}
           \dot{\xi}_g \\
           \dot{\eta}_g \\
           \dot{\theta}_g \\
           \dot{\xi}_a \\
           \dot{\eta}_a \\
           \dot{\theta}_a
         \end{bmatrix} 
         =  \begin{bmatrix}
           \dot{x}_1 \\
           \dot{x}_2 \\
           \dot{x}_3 \\
           \dot{x}_4 \\
           \dot{x}_5 \\
           \dot{x}_6
         \end{bmatrix} 
         = \begin{bmatrix}
           \mathcal{F}_1(x,u) \\
           \mathcal{F}_2(x,u) \\
           \mathcal{F}_3(x,u) \\
           \mathcal{F}_4(x,u) \\
           \mathcal{F}_5(x,u) \\
           \mathcal{F}_6(x,u) \\
         \end{bmatrix}
         = \begin{bmatrix}
           u_1 \cos{x_3} \\
           u_1 \sin{x_3} \\
           \frac{u_1}{L} \tan{u_2} \\
           u_3 \cos{x_6} \\
           u_3 \sin{x_6} \\
           u_4 \\
         \end{bmatrix} 
\end{align*}


\begin{align*}
  y &= \begin{bmatrix} \arctan{(\frac{x_5 - x_2}{x_4 - x_1}}) - x_3   \\ 
                \sqrt{ (x_1 - x_4)^2 + (x_2 - x_5)^2}  \\
                \arctan{(\frac{x_2 - x_5}{x_1 - x_4})} - x_6 \\
                x_4 \\
                x_5
  \end{bmatrix} 
            = \begin{bmatrix}
           \mathcal{H}_1(x,u) \\
           \mathcal{H}_2(x,u) \\
           \mathcal{H}_3(x,u) \\
           \mathcal{H}_4(x,u) \\
           \mathcal{H}_5(x,u) \\
         \end{bmatrix}\\
\end{align*}

We now need to compute the partials of  $\mathcal{F}_{1\dots6}$ with respect to x,

\[ \begin{matrix*}[l]
\frac{\partial\mathcal{F}_1}{\partial x_1} = 0 &
\frac{\partial\mathcal{F}_2}{\partial x_1} = 0 & 
\frac{\partial\mathcal{F}_3}{\partial x_1} = 0 & 
\frac{\partial\mathcal{F}_4}{\partial x_1} = 0 & 
\frac{\partial\mathcal{F}_5}{\partial x_1} = 0 &  
\frac{\partial\mathcal{F}_6}{\partial x_1} = 0 & \\ \\
\frac{\partial\mathcal{F}_1}{\partial x_2} = 0 &
\frac{\partial\mathcal{F}_2}{\partial x_2} = 0 & 
\frac{\partial\mathcal{F}_3}{\partial x_2} = 0 & 
\frac{\partial\mathcal{F}_4}{\partial x_2} = 0 & 
\frac{\partial\mathcal{F}_5}{\partial x_2} = 0 &  
\frac{\partial\mathcal{F}_6}{\partial x_2} = 0 & \\ \\
\frac{\partial\mathcal{F}_1}{\partial x_3} = -u_1\sin{x_3} &
\frac{\partial\mathcal{F}_2}{\partial x_3} = u_1\cos{x_3} & 
\frac{\partial\mathcal{F}_3}{\partial x_3} = 0 & 
\frac{\partial\mathcal{F}_4}{\partial x_3} = 0 & 
\frac{\partial\mathcal{F}_5}{\partial x_3} = 0 &  
\frac{\partial\mathcal{F}_6}{\partial x_3} = 0 & \\ \\
\frac{\partial\mathcal{F}_1}{\partial x_4} = 0 &
\frac{\partial\mathcal{F}_2}{\partial x_4} = 0 & 
\frac{\partial\mathcal{F}_3}{\partial x_4} = 0 & 
\frac{\partial\mathcal{F}_4}{\partial x_4} = 0 & 
\frac{\partial\mathcal{F}_5}{\partial x_4} = 0 &  
\frac{\partial\mathcal{F}_6}{\partial x_4} = 0 & \\ \\
\frac{\partial\mathcal{F}_1}{\partial x_5} = 0 &
\frac{\partial\mathcal{F}_2}{\partial x_5} = 0 & 
\frac{\partial\mathcal{F}_3}{\partial x_5} = 0 & 
\frac{\partial\mathcal{F}_4}{\partial x_5} = 0 & 
\frac{\partial\mathcal{F}_5}{\partial x_5} = 0 &  
\frac{\partial\mathcal{F}_6}{\partial x_5} = 0 & \\ \\
\frac{\partial\mathcal{F}_1}{\partial x_6} = 0 &
\frac{\partial\mathcal{F}_2}{\partial x_6} = 0 & 
\frac{\partial\mathcal{F}_3}{\partial x_6} = 0 & 
\frac{\partial\mathcal{F}_4}{\partial x_6} = -u_3\sin{x_6} & 
\frac{\partial\mathcal{F}_5}{\partial x_6} = u_3\cos{x_6} &  
\frac{\partial\mathcal{F}_6}{\partial x_6} = 0 & \\ \\
\end{matrix*}\]

and with respect to u,

\[ \begin{matrix*}[l]
\frac{\partial\mathcal{F}_1}{\partial u_1} = \cos{x_3} &
\frac{\partial\mathcal{F}_2}{\partial u_1} = \sin{x_3} & 
\frac{\partial\mathcal{F}_3}{\partial u_1} = \frac{\tan{u_2}}{L} & 
\frac{\partial\mathcal{F}_4}{\partial u_1} = 0 & 
\frac{\partial\mathcal{F}_5}{\partial u_1} = 0 &  
\frac{\partial\mathcal{F}_6}{\partial u_1} = 0 & \\ \\
\frac{\partial\mathcal{F}_1}{\partial u_2} = 0 &
\frac{\partial\mathcal{F}_2}{\partial u_2} = 0 & 
\frac{\partial\mathcal{F}_3}{\partial u_2} = \frac{u_1}{L}\sec^2{u_2} & 
\frac{\partial\mathcal{F}_4}{\partial u_2} = 0 & 
\frac{\partial\mathcal{F}_5}{\partial u_2} = 0 &  
\frac{\partial\mathcal{F}_6}{\partial u_2} = 0 & \\ \\
\frac{\partial\mathcal{F}_1}{\partial u_3} = 0 &
\frac{\partial\mathcal{F}_2}{\partial u_3} = 0 & 
\frac{\partial\mathcal{F}_3}{\partial u_3} = 0 & 
\frac{\partial\mathcal{F}_4}{\partial u_3} = \cos{x_6} & 
\frac{\partial\mathcal{F}_5}{\partial u_3} = \sin{x_6} &  
\frac{\partial\mathcal{F}_6}{\partial u_3} = 0 & \\ \\
\frac{\partial\mathcal{F}_1}{\partial u_4} = 0 &
\frac{\partial\mathcal{F}_2}{\partial u_4} = 0 & 
\frac{\partial\mathcal{F}_3}{\partial u_4} = 0 & 
\frac{\partial\mathcal{F}_4}{\partial u_4} = 0 & 
\frac{\partial\mathcal{F}_5}{\partial u_4} = 0 &  
\frac{\partial\mathcal{F}_6}{\partial u_4} = 1 & \\ \\
\end{matrix*}\]

finally we compute $\mathcal{H}_{1 \dots 5}$ with respect to x.  In the following we show the two most complex partial derivative computations. The remaining partials were computed using similar techniques and therefore we omit them for brevity.  

Utilize the chain rule with $u=(\frac{x_5 - x_2}{x_4 - x_1})$, then, 
\\
\begin{align*}
    \frac{\partial{\mathcal{H}_1}}{\partial{x_1}} &= \frac{\partial{\mathcal{H}_1}}{\partial{u}} \times \frac{\partial{u}}{\partial{x_1}} \\
    &= \bigg( \frac{1}{(\frac{x_5 - x_2}{x_4 - x_1})^2 + 1} \bigg) \times \bigg(  \frac{0 \times (x_4-x_1) - (x_5-x_2)\times -1 }{(x_4 - x_1)^2}  \bigg) \\
    &= \bigg( \frac{1}{(\frac{x_5 - x_2}{x_4 - x_1})^2 + 1} \bigg) \times \bigg(  \frac{(x_5-x_2)}{(x_4 - x_1)^2}  \bigg) \\
    &= \bigg( \frac{1}{\frac{(x_5 - x_2)^2}{(x_4 - x_1)^2} + 1} \bigg) \times \bigg(  \frac{(x_5-x_2)}{(x_4 - x_1)^2}  \bigg) \\
    &= \bigg( \frac{1}{\frac{(x_5 - x_2)^2 + (x_4-x_1)^2}{(x_4 - x_1)^2}} \bigg) \times \bigg(  \frac{(x_5-x_2)}{(x_4 - x_1)^2}  \bigg) \\
    &= \bigg( \frac{(x_4 - x_1)^2}{(x_5 - x_2)^2 + (x_4-x_1)^2} \bigg) \times \bigg(  \frac{(x_5-x_2)}{(x_4 - x_1)^2}  \bigg) \\
    &= \frac{x_5-x_2}{(x_5 - x_2)^2 + (x_4-x_1)^2}
\end{align*}
\\
and we show the case when the partial we are computing is in the numerator,
\\
\begin{align*}
    \frac{\partial{\mathcal{H}_1}}{\partial{x_2}} &= \frac{\partial{\mathcal{H}_1}}{\partial{u}} \times \frac{\partial{u}}{\partial{x_2}} \\
    &= \bigg( \frac{1}{(\frac{x_5 - x_2}{x_4 - x_1})^2 + 1} \bigg) \times \bigg(  \frac{-1 \times (x_4-x_1) - (x_5-x_2)\times 0 }{(x_4 - x_1)^2}  \bigg) \\
    &= \bigg( \frac{1}{(\frac{x_5 - x_2}{x_4 - x_1})^2 + 1} \bigg) \times \bigg(  -\frac{(x_4-x_1)}{(x_4 - x_1)^2}  \bigg) \\
    &= \bigg( \frac{1}{\frac{(x_5 - x_2)^2}{(x_4 - x_1)^2} + 1} \bigg) \times \bigg(  -\frac{(x_4-x_1)}{(x_4 - x_1)^2}  \bigg) \\
    &= \bigg( \frac{1}{\frac{(x_5 - x_2)^2 + (x_4-x_1)^2}{(x_4 - x_1)^2}} \bigg) \times \bigg(  -\frac{(x_4-x_1)}{(x_4 - x_1)^2}  \bigg) \\
    &= \bigg( \frac{(x_4 - x_1)^2}{(x_5 - x_2)^2 + (x_4-x_1)^2} \bigg) \times \bigg(  -\frac{(x_4-x_1)}{(x_4 - x_1)^2}  \bigg) \\
    &= -\frac{x_4-x_1}{(x_5 - x_2)^2 + (x_4-x_1)^2}
\end{align*}
\\
Now, let $u=(x_1-x_4)^2 + (x_2-x_5)^2$ , $v=(x_1-x_4)^2$, and $w=(x_1-x_4)$, then for our next complex partial we have,
\\
\begin{align*}
    \frac{\partial{\mathcal{H}_1}}{\partial{x_1}} &= \frac{\partial{\mathcal{H}_1}}{\partial{u}} \times \frac{\partial{u}}{\partial{v}}\times \frac{\partial{v}}{\partial{w}} \times \frac{\partial{w}}{\partial{x_1}} \\
    &=  \bigg( \frac{1}{2} \times \frac{1}{\sqrt{u} } \bigg) \times (1) \times \bigg( 2(w)\bigg) \times(1) \\
    &=  \bigg( \frac{1}{2} \times \frac{1}{\sqrt{(x_1-x_4)^2+(x_2-x_5)^2} } \bigg) \times (1) \times \bigg( 2(x_1-x_4)\bigg) \times(1) \\
    &= \frac{x_1 - x_4}{\sqrt{(x_1-x_4)^2 + (x_2-x_5)^2}}
\end{align*}
\\

therefore we have the following Jacobians,

\begin{align*}
    \frac{\partial f}{\partial x} &= 
    \begin{bmatrix} 0 & 0 & -u_1\sin x_3 & 0 & 0 & 0 \\ 
                    0 & 0 & u_1\cos x_3 & 0 & 0 & 0 \\ 
                    0 & 0 & 0 & 0 & 0 & 0 \\ 
                    0 & 0 & 0 & 0 & 0 & u_3\sin x_6 \\ 
                    0 & 0 & 0 & 0 & 0 & -u_3 \cos x_6 \\ 
                    0 & 0 & 0 & 0 & 0 & 0 \end{bmatrix} \\
    \frac{\partial f}{\partial u} &= 
      \begin{bmatrix} \cos x_3 & 0 & 0 & 0 \\ 
      \sin x_3 & 0 & 0 & 0 \\ 
      \frac{\tan u_2}{L}  & \frac{u_1\sec^2u_2}{L} & 0 & 0 \\ 
       0 & 0 & \cos x_6 & 0 \\ 
       0 & 0 & \sin x_6 & 0 \\ 
       0 & 0 & 0 & 1 
       \end{bmatrix} \\
    \frac{\partial h}{\partial x} &= 
    \begin{bmatrix} \frac{x_5-x_2}{(x_4-x_1)^2+(x_2-x_5)^2} & -\frac{x_4-x_1}{(x_4-x_1)^2+(x_2-x_5)^2} & -1 & -\frac{x_5-x_2}{(x_4-x_1)^2+(x_5-x_2)^2} & \frac{x_4-x_1}{(x_4-x_1)^2+(x_5-x_2)^2} & 0 \\
    \frac{x_1+x_4}{\sqrt{(x_1+x_4)^2+(x_2+x_5)^2}} & \frac{x_2+x_5}{\sqrt{(x_1+x_4)^2+(x_2+x_5)^2}} & 0 & \frac{x_1 + x_4}{\sqrt{(x_1+x_4)^2+(x_2+x_5)^2}} & \frac{x_2+x_5}{\sqrt{(x_1+x_4)^2+(x_2+x_5)^2}} & 0 \\ 
    -\frac{x_2-x_5}{(x_1-x_4)^2+(x_2-x_5)^2} & \frac{x_1-x_4}{(x_1-x_4)^2+(x_2-x_5)^2} & 0 & \frac{x_2-x_5}{(x_1-x_4)^2+(x_2-x_5)^2} & 
    -\frac{x_1-x_4}{(x_1-x_4)^2+(x_2-x_5)^2} & -1 \\
    0 & 0 & 0 & 1 & 0 & 0 \\ 0 & 0 & 0 & 0 & 1 & 0 \end{bmatrix} \\
    \frac{\partial h}{\partial u} &= 
      \begin{bmatrix} 0 & 0 & 0 & 0  \\ 
                      0 & 0 & 0 & 0  \\ 
                      0 & 0 & 0 & 0  \\ 
                      0 & 0 & 0 & 0  \\ 
                      0 & 0 & 0 & 0  \\ 
                     \end{bmatrix} \\
\end{align*}




\end{framed}


\textbf{\\ Part II: Stochastic Nonlinear Filtering }
\begin{framed}
   
\end{framed}


\end{document}