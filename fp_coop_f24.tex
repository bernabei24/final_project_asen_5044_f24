  %%%%%%%%%%%%%%%%%%%%%%%%%%%%%%%%%%%%%%%%%

% Short Sectioned Assignment
% LaTeX Template
% Version 1.0 (5/5/12)
%
% This template has been downloaded from:
% http://www.LaTeXTemplates.com
%
% Original author:
% Frits Wenneker (http://www.howtotex.com)
%
% License:
% CC BY-NC-SA 3.0 (http://creativecommons.org/licenses/by-nc-sa/3.0/)
%
%%%%%%%%%%%%%%%%%%%%%%%%%%%%%%%%%%%%%%%%%

%----------------------------------------------------------------------------------------
%   PACKAGES AND OTHER DOCUMENT CONFIGURATIONS
%----------------------------------------------------------------------------------------

\documentclass[paper=a4, fontsize=11pt]{scrartcl} % A4 paper and 11pt font size

\usepackage[T1]{fontenc} % Use 8-bit encoding that has 256 glyphs
\usepackage{palatino} % Use the Adobe Utopia font for the document - comment this line to return to the LaTeX default
\usepackage[english]{babel} % English language/hyphenation
\usepackage{amsmath,amsfonts,amsthm} % Math packages
\usepackage{multicol,lastpage,fullpage,framed,fancybox,enumerate,tikz}
\usepackage{lipsum} % Used for inserting dummy 'Lorem ipsum' text into the template
\usepackage{mathrsfs}
\usepackage{graphicx}
\usepackage{mathtools}
%
%
%
%   A T T E N T I O N ! ! !
%
%   SET YOUR GRAPHICS FOLDER IN THE LINE BELOW
%
\graphicspath{ {.} }
\usepackage{sectsty} % Allows customizing section commands
\allsectionsfont{\centering \normalfont\scshape} % Make all sections centered, the default font and small caps

\usepackage{fancyhdr} % Custom headers and footers
\pagestyle{fancy plain} % Makes all pages in the document conform to the custom headers and footers
\fancyhead[L]{\textsc{CSEN 5830}}
\fancyhead[R]{\textsc{Homework #3}} % No page header - if you want one, create it in the same way as the footers below
\fancyfoot[L]{} % Empty left footer
\fancyfoot[C]{} % Empty center footer
\fancyfoot[C]{- \thepage -} % Page numbering for right footer
\renewcommand{\headrulewidth}{0.5pt} % Remove header underlines
\renewcommand{\footrulewidth}{0pt} % Remove footer underlines
\setlength{\headheight}{13.6pt} % Customize the height of the header

\fancypagestyle{noheader}{    %create style that allows to skip header manually on pages with new section
    \fancyhead{}
    \renewcommand{\headrulewidth}{0pt}
}

\numberwithin{equation}{section} % Number equations within sections (i.e. 1.1, 1.2, 2.1, 2.2 instead of 1, 2, 3, 4)
\numberwithin{figure}{section} % Number figures within sections (i.e. 1.1, 1.2, 2.1, 2.2 instead of 1, 2, 3, 4)
\numberwithin{table}{section} % Number tables within sections (i.e. 1.1, 1.2, 2.1, 2.2 instead of 1, 2, 3, 4)

\setlength\parindent{0pt} % Removes all indentation from paragraphs - comment this line for an assignment with lots of text

%----------------------------------------------------------------------------------------
%   TITLE SECTION
%----------------------------------------------------------------------------------------

\newcommand{\horrule}[1]{\rule{\linewidth}{#1}} % Create horizontal rule command with 1 argument of height

\title{
\normalfont \LARGE
\textsc{University of Colorado at Boulder} \\ [25pt] % Your university, school and/or department name(s)
\textsc{ASEN 5044 - Statistical State Estimation for Dynamical Systems} \\ [20pt]
\textsc{Fall 2024} \\ [20pt]
\textsc{Professor: Dr. Nisar Ahmed} \\ [12pt]
\horrule{1pt} \\[0.4cm] % Thin top horizontal rule
\huge Final Project \\ % The assignment title
\huge (Cooperative Air-Ground Robot Localization) \\ 
\horrule{1pt} \\[0.6cm] % Thick bottom horizontal rule
}

\author{
  \textsc{ Team Members:} \\ [4 mm]
  \textsc{ Nicholas Martinez}\\[2mm]
  \textsc{ Whit Whittall } \\[2mm]
  \textsc{ Michael Bernabei}\\[2mm]
}

\date{\normalsize\today} % Today's date or a custom date

\begin{document}

\maketitle % Print the title
\thispagestyle{empty} %make title page header empty
\newpage


%----------------------------------------------------------------------------------------
%   PROBLEM SECTION
%----------------------------------------------------------------------------------------
%

\textbf{\\ Part I: Deterministic System Analysis }
\begin{framed}
\textbf{Part 1.} \\

We are given the Equation Of Motion (EOM) for the Unmanned Ground Vehicle (UGV).  The EOM is,

\begin{align*}
    \dot{\xi_g} &= v_g \cos{\theta_g} + \Tilde{w}_{x,g} \\
    \dot{\eta_g} &= v_g \sin{\theta_g} + \Tilde{w}_{y,g} \\
    \dot{\theta_g} &= \frac{v_g}{L}\tan{\phi_g}  + \Tilde{w}_{\omega,g} \\
\end{align*}

and for the Unmanned Aerial Vehicle (UAV) we have the following EOM,

\begin{align*}
    \dot{\xi_a} &= v_a \cos{\theta_a} + \Tilde{w}_{x,a} \\
    \dot{\eta_a} &= v_a \sin{\theta_a} + \Tilde{w}_{y,a} \\
    \dot{\theta_a} &= \frac{v_a}{L}\tan{\phi_a}  + \Tilde{w}_{\omega,a} \\
\end{align*}

where $\Tilde{w}_a = [ \Tilde{w}_{x,a},  \Tilde{w}_{y,a},  \Tilde{w}_{\omega,a}]^T$ and $\Tilde{w}_g = [ \Tilde{w}_{x,g},  \Tilde{w}_{y,g},  \Tilde{w}_{\omega,g}]^T$ are the process noise for the UAV And UGV respectively.  We are also given the following sensing model,

\begin{align*}
y(t) &=  
\begin{bmatrix} \arctan{(\frac{\eta_a - \eta_g}{\xi_a - \xi_g}}) - \theta_g   \\ 
                \sqrt{  (\xi_g - \xi_a)^2 + (\eta_g - \eta_a)^2}  \\
                \arctan{(\frac{\eta_g - \eta_a}{\xi_g - \xi_a})} - \theta_a \\
                \xi_a \\
                \eta_a
\end{bmatrix}  + \Tilde{\bold{v}}(t) \\
\end{align*}

where $\Tilde{\bold{v}}(t) \in \mathbb{R}^5$ is the sensor error vector.  Finally, we are given the combined states, control inputs, and disturbance inputs as,

\begin{align*}
    \bold{x}(t) &= [ \xi_g \ \ \ \eta_g \ \ \ \theta_g \ \ \ \xi_a \ \ \ \eta_a \ \ \ \theta_a ] ^T ,\\
    \bold{u}(t) &= [ \bold{u}_g \ \ \bold{u}_a ]^T ,\\
    \bold{\Tilde{w}}(t) &= [\bold{\Tilde{w}_g \ \ \bold{\Tilde{w}}_a}]^T \\
\end{align*}

The state is $x = [\xi_g \ \  \eta_g \ \  \theta_g \ \  \xi_a \ \  \eta_a \ \ \theta_a ]^T = [ x_1 \ \ x_2 \  x_3 \ \ x_4 \ \ x_5 \ \ x_6 ]^T$ and our inputs $u = [ \bold{u}_g \ \ \bold{u}_a ]^T = [v_g \ \ \phi_g \ \ v_a \ \ \phi_a]^T = [u_1 \ \ u_2 \ \ u_3 \ \ u_4 ]^T$.  We then have the following after substituting in our state and input variables,

\begin{align*}
    \dot{x} &= \begin{bmatrix}
           \dot{\xi}_g \\
           \dot{\eta}_g \\
           \dot{\theta}_g \\
           \dot{\xi}_a \\
           \dot{\eta}_a \\
           \dot{\theta}_a
         \end{bmatrix} 
         =  \begin{bmatrix}
           \dot{x}_1 \\
           \dot{x}_2 \\
           \dot{x}_3 \\
           \dot{x}_4 \\
           \dot{x}_5 \\
           \dot{x}_6
         \end{bmatrix} 
         = \begin{bmatrix}
           \mathcal{F}_1(x,u) \\
           \mathcal{F}_2(x,u) \\
           \mathcal{F}_3(x,u) \\
           \mathcal{F}_4(x,u) \\
           \mathcal{F}_5(x,u) \\
           \mathcal{F}_6(x,u) \\
         \end{bmatrix}
         = \begin{bmatrix}
           u_1 \cos{x_3} \\
           u_1 \sin{x_3} \\
           \frac{u_1}{L} \tan{u_2} \\
           u_3 \cos{x_6} \\
           u_3 \sin{x_6} \\
           \frac{u_3}{L} \tan{u_4} \\
         \end{bmatrix} 
\end{align*}


\begin{align*}
  y &= \begin{bmatrix} \arctan{(\frac{x_5 - x_2}{x_4 - x_1}}) - x_3   \\ 
                \sqrt{ (x_1 - x_4)^2 + (x_2 - x_5)^2}  \\
                \arctan{(\frac{x_2 - x_5}{x_1 - x_4})} - x_6 \\
                x_4 \\
                x_5
  \end{bmatrix}  \\
\end{align*}

We now need to compute the partials with respect to x,

\[ \begin{matrix*}[l]
\frac{\partial\mathcal{F}_1}{\partial x_1} = 0 &
\frac{\partial\mathcal{F}_2}{\partial x_1} = 0 & 
\frac{\partial\mathcal{F}_3}{\partial x_1} = 0 & 
\frac{\partial\mathcal{F}_4}{\partial x_1} = 0 & 
\frac{\partial\mathcal{F}_5}{\partial x_1} = 0 &  
\frac{\partial\mathcal{F}_6}{\partial x_1} = 0 & \\ \\
\frac{\partial\mathcal{F}_1}{\partial x_2} = 0 &
\frac{\partial\mathcal{F}_2}{\partial x_2} = 0 & 
\frac{\partial\mathcal{F}_3}{\partial x_2} = 0 & 
\frac{\partial\mathcal{F}_4}{\partial x_2} = 0 & 
\frac{\partial\mathcal{F}_5}{\partial x_2} = 0 &  
\frac{\partial\mathcal{F}_6}{\partial x_2} = 0 & \\ \\
\frac{\partial\mathcal{F}_1}{\partial x_3} = -u_1\sin{x_3} &
\frac{\partial\mathcal{F}_2}{\partial x_3} = u_1\cos{x_3} & 
\frac{\partial\mathcal{F}_3}{\partial x_3} = 0 & 
\frac{\partial\mathcal{F}_4}{\partial x_3} = 0 & 
\frac{\partial\mathcal{F}_5}{\partial x_3} = 0 &  
\frac{\partial\mathcal{F}_6}{\partial x_3} = 0 & \\ \\
\frac{\partial\mathcal{F}_1}{\partial x_4} = 0 &
\frac{\partial\mathcal{F}_2}{\partial x_4} = 0 & 
\frac{\partial\mathcal{F}_3}{\partial x_4} = 0 & 
\frac{\partial\mathcal{F}_4}{\partial x_4} = 0 & 
\frac{\partial\mathcal{F}_5}{\partial x_4} = 0 &  
\frac{\partial\mathcal{F}_6}{\partial x_4} = 0 & \\ \\
\frac{\partial\mathcal{F}_1}{\partial x_5} = 0 &
\frac{\partial\mathcal{F}_2}{\partial x_5} = 0 & 
\frac{\partial\mathcal{F}_3}{\partial x_5} = 0 & 
\frac{\partial\mathcal{F}_4}{\partial x_5} = 0 & 
\frac{\partial\mathcal{F}_5}{\partial x_5} = 0 &  
\frac{\partial\mathcal{F}_6}{\partial x_5} = 0 & \\ \\
\frac{\partial\mathcal{F}_1}{\partial x_6} = 0 &
\frac{\partial\mathcal{F}_2}{\partial x_6} = 0 & 
\frac{\partial\mathcal{F}_3}{\partial x_6} = 0 & 
\frac{\partial\mathcal{F}_4}{\partial x_6} = -u_3\sin{x_6} & 
\frac{\partial\mathcal{F}_5}{\partial x_6} = u_3\cos{x_6} &  
\frac{\partial\mathcal{F}_6}{\partial x_6} = 0 & \\ \\
\end{matrix*}\]

and with respect to u,

\[ \begin{matrix*}[l]
\frac{\partial\mathcal{F}_1}{\partial u_1} = \cos{x_3} &
\frac{\partial\mathcal{F}_2}{\partial u_1} = \sin{x_3} & 
\frac{\partial\mathcal{F}_3}{\partial u_1} = \frac{\tan{u_2}}{L} & 
\frac{\partial\mathcal{F}_4}{\partial u_1} = 0 & 
\frac{\partial\mathcal{F}_5}{\partial u_1} = 0 &  
\frac{\partial\mathcal{F}_6}{\partial u_1} = 0 & \\ \\
\frac{\partial\mathcal{F}_1}{\partial u_2} = 0 &
\frac{\partial\mathcal{F}_2}{\partial u_2} = 0 & 
\frac{\partial\mathcal{F}_3}{\partial u_2} = \frac{u_1}{L}\sec^2{u_2} & 
\frac{\partial\mathcal{F}_4}{\partial u_2} = 0 & 
\frac{\partial\mathcal{F}_5}{\partial u_2} = 0 &  
\frac{\partial\mathcal{F}_6}{\partial u_2} = 0 & \\ \\
\frac{\partial\mathcal{F}_1}{\partial u_3} = 0 &
\frac{\partial\mathcal{F}_2}{\partial u_3} = 0 & 
\frac{\partial\mathcal{F}_3}{\partial u_3} = 0 & 
\frac{\partial\mathcal{F}_4}{\partial u_3} = \cos{x_6} & 
\frac{\partial\mathcal{F}_5}{\partial u_3} = \sin{x_6} &  
\frac{\partial\mathcal{F}_6}{\partial u_3} = \frac{\tan{u_4}}{L} & \\ \\
\frac{\partial\mathcal{F}_1}{\partial u_4} = 0 &
\frac{\partial\mathcal{F}_2}{\partial u_4} = 0 & 
\frac{\partial\mathcal{F}_3}{\partial u_4} = 0 & 
\frac{\partial\mathcal{F}_4}{\partial u_4} = 0 & 
\frac{\partial\mathcal{F}_5}{\partial u_4} = 0 &  
\frac{\partial\mathcal{F}_6}{\partial u_4} = \frac{u_3}{L}\sec^2{u_4} & \\ \\
\end{matrix*}\]


% \begin{aligned}
% \frac{\partial\mathcal{F}_1}{\partial x_1} = 0; \ \ 
% \frac{\partial\mathcal{F}_2}{\partial x_1} = 0; \ \  
% \frac{\partial\mathcal{F}_3}{\partial x_1} = 0; \ \  
% \frac{\partial\mathcal{F}_4}{\partial x_1} = 0; \ \  
% \frac{\partial\mathcal{F}_5}{\partial x_1} = 0; \ \  
% \frac{\partial\mathcal{F}_6}{\partial x_1} = 0; \ \ 
% \\
% \frac{\partial\mathcal{F}_1}{\partial x_2} = 0; \ \ 
% \frac{\partial\mathcal{F}_2}{\partial x_2} = 0; \ \  
% \frac{\partial\mathcal{F}_3}{\partial x_2} = 0; \ \  
% \frac{\partial\mathcal{F}_4}{\partial x_2} = 0; \ \  
% \frac{\partial\mathcal{F}_5}{\partial x_2} = 0; \ \  
% \frac{\partial\mathcal{F}_6}{\partial x_2} = 0; \ \ 
% \\
% \frac{\partial\mathcal{F}_1}{\partial x_3} = -u_1\sin{x_3}; \ \ 
% \frac{\partial\mathcal{F}_2}{\partial x_3} = u_1\cos{x_3}; \ \  
% \frac{\partial\mathcal{F}_3}{\partial x_3} = 0; \ \  
% \frac{\partial\mathcal{F}_4}{\partial x_3} = 0; \ \  
% \frac{\partial\mathcal{F}_5}{\partial x_3} = 0; \ \  
% \frac{\partial\mathcal{F}_6}{\partial x_3} = 0; \ \ 
% \\
% \frac{\partial\mathcal{F}_1}{\partial x_4} = 0; \ \ 
% \frac{\partial\mathcal{F}_2}{\partial x_4} = 0; \ \  
% \frac{\partial\mathcal{F}_3}{\partial x_4} = 0; \ \  
% \frac{\partial\mathcal{F}_4}{\partial x_4} = 0; \ \  
% \frac{\partial\mathcal{F}_5}{\partial x_4} = 0; \ \  
% \frac{\partial\mathcal{F}_6}{\partial x_4} = 0; \ \ 
% \\
% \frac{\partial\mathcal{F}_1}{\partial x_5} = 0; \ \ 
% \frac{\partial\mathcal{F}_2}{\partial x_5} = 0; \ \  
% \frac{\partial\mathcal{F}_3}{\partial x_5} = 0; \ \  
% \frac{\partial\mathcal{F}_4}{\partial x_5} = 0; \ \  
% \frac{\partial\mathcal{F}_5}{\partial x_5} = 0; \ \  
% \frac{\partial\mathcal{F}_6}{\partial x_5} = 0; \ \ 
% \\
% \frac{\partial\mathcal{F}_1}{\partial x_6} = 0; \ \ 
% \frac{\partial\mathcal{F}_2}{\partial x_6} = 0; \ \  
% \frac{\partial\mathcal{F}_3}{\partial x_6} = 0; \ \  
% \frac{\partial\mathcal{F}_4}{\partial x_6} = 0; \ \  
% \frac{\partial\mathcal{F}_5}{\partial x_6} = 0; \ \  
% \frac{\partial\mathcal{F}_6}{\partial x_6} = 0; \ \ 
% \\
% \end{align*}


\end{framed}


\textbf{\\ Part II: Stochastic Nonlinear Filtering }
\begin{framed}
   
\end{framed}


\end{document}